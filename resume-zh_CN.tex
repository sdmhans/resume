% !TEX TS-program = xelatex
% !TEX encoding = UTF-8 Unicode
% !Mode:: "TeX:UTF-8"

\documentclass{resume}
\usepackage{zh_CN-Adobefonts_external} % Simplified Chinese Support using external fonts (./fonts/zh_CN-Adobe/)
%\usepackage{zh_CN-Adobefonts_internal} % Simplified Chinese Support using system fonts
\usepackage{linespacing_fix} % disable extra space before next section
\usepackage{cite}

\begin{document}
\pagenumbering{gobble} % suppress displaying page number

\name{盛栋铭}

\basicInfo{
  \email{201530612699@mail.scut.edu.cn} \textperiodcentered\ 
  \phone{(+86) 135-333-45971} \textperiodcentered\ 
  \github[sdmhans]{https://github.com/sdmhans}}
 
\section{\faGraduationCap\ 教育背景}
\datedsubsection{\textbf{华南理工大学}, 广州}{2015年9月 -- 2019年6月}
\textit{在读本科生}\ 软件工程卓越班, GPA: 3.8 / 4.0 (排名8 / 42)

\section{\faUsers\ 项目经历}
\datedsubsection{\textbf{Kaggle 冰山/船只识别比赛(Statoil Iceberg Classifier Challenge)}}{2018年1月}

\begin{onehalfspacing}
判断所给卫星图像中的是冰山还是船只(二分类图像识别问题)
\begin{itemize}
  \item 单人参赛,历时3周,排名:223 / 3343 (前7\%,Kaggle铜牌)
  \item 学习使用keras框架搭建CNN模型
  \item 构建了完整的工作流(图像预处理-模型搭建和训练-结果预测-模型融合)
  \item 采用测试时增强(TTA)和一种简单的伪标签方法提升模型效果
\end{itemize}
\end{onehalfspacing}

\datedsubsection{\textbf{2017 CCF大数据与计算智能大赛(BDCI-2017)\\“城市自行车的出行行为分析及效率优化”赛题}}{2017年10月 -- 2017年12月}
\begin{onehalfspacing}
给出400个站点前8个月每天的自行车借/还量,预测后2个月每天的自行车借/还量(时序预测问题)
\begin{itemize}
  \item 5人组队参赛,历时2个月,排名:2 / 78
  \item 从天气、节假日、站点地理位置等角度提取100+的人工特征
\end{itemize}
\end{onehalfspacing}

% Reference Test
%\datedsubsection{\textbf{Paper Title\cite{zaharia2012resilient}}}{May. 2015}
%An xxx optimized for xxx\cite{verma2015large}
%\begin{itemize}
%  \item main contribution
%\end{itemize}

\section{\faHeartO\ 奖项和证书}
\datedline{大学英语四级证书(总分:612)}{2015 年12 月}
\datedline{大学英语六级证书(总分:632)}{2016 年$\ \;$6  月}
\datedline{华南理工大学2015-2016学年度校三等奖学金}{2016 年12 月}
\datedline{华南理工大学2016-2017学年度校一等奖学金}{2017 年12 月}
\datedline{浙江大学计算机程序设计能力考试(甲级)成绩证书(总分:97 / 100)}{2018 年$\ \;$3 月}


\section{\faInfo\ 其他}
% increase linespacing [parsep=0.5ex]
\begin{itemize}[parsep=0.5ex]
  \item 编程语言:python、C++
  \item 感兴趣的方向:使用深度学习解决计算机视觉问题
  \item 特长爱好:乒乓球、钢琴
\end{itemize}

%% Reference
%\newpage
%\bibliographystyle{IEEETran}
%\bibliography{mycite}
\end{document}
